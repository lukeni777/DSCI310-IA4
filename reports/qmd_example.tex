% Options for packages loaded elsewhere
\PassOptionsToPackage{unicode}{hyperref}
\PassOptionsToPackage{hyphens}{url}
\PassOptionsToPackage{dvipsnames,svgnames,x11names}{xcolor}
%
\documentclass[
  letterpaper,
  DIV=11,
  numbers=noendperiod]{scrartcl}

\usepackage{amsmath,amssymb}
\usepackage{iftex}
\ifPDFTeX
  \usepackage[T1]{fontenc}
  \usepackage[utf8]{inputenc}
  \usepackage{textcomp} % provide euro and other symbols
\else % if luatex or xetex
  \usepackage{unicode-math}
  \defaultfontfeatures{Scale=MatchLowercase}
  \defaultfontfeatures[\rmfamily]{Ligatures=TeX,Scale=1}
\fi
\usepackage{lmodern}
\ifPDFTeX\else  
    % xetex/luatex font selection
\fi
% Use upquote if available, for straight quotes in verbatim environments
\IfFileExists{upquote.sty}{\usepackage{upquote}}{}
\IfFileExists{microtype.sty}{% use microtype if available
  \usepackage[]{microtype}
  \UseMicrotypeSet[protrusion]{basicmath} % disable protrusion for tt fonts
}{}
\makeatletter
\@ifundefined{KOMAClassName}{% if non-KOMA class
  \IfFileExists{parskip.sty}{%
    \usepackage{parskip}
  }{% else
    \setlength{\parindent}{0pt}
    \setlength{\parskip}{6pt plus 2pt minus 1pt}}
}{% if KOMA class
  \KOMAoptions{parskip=half}}
\makeatother
\usepackage{xcolor}
\setlength{\emergencystretch}{3em} % prevent overfull lines
\setcounter{secnumdepth}{-\maxdimen} % remove section numbering
% Make \paragraph and \subparagraph free-standing
\makeatletter
\ifx\paragraph\undefined\else
  \let\oldparagraph\paragraph
  \renewcommand{\paragraph}{
    \@ifstar
      \xxxParagraphStar
      \xxxParagraphNoStar
  }
  \newcommand{\xxxParagraphStar}[1]{\oldparagraph*{#1}\mbox{}}
  \newcommand{\xxxParagraphNoStar}[1]{\oldparagraph{#1}\mbox{}}
\fi
\ifx\subparagraph\undefined\else
  \let\oldsubparagraph\subparagraph
  \renewcommand{\subparagraph}{
    \@ifstar
      \xxxSubParagraphStar
      \xxxSubParagraphNoStar
  }
  \newcommand{\xxxSubParagraphStar}[1]{\oldsubparagraph*{#1}\mbox{}}
  \newcommand{\xxxSubParagraphNoStar}[1]{\oldsubparagraph{#1}\mbox{}}
\fi
\makeatother


\providecommand{\tightlist}{%
  \setlength{\itemsep}{0pt}\setlength{\parskip}{0pt}}\usepackage{longtable,booktabs,array}
\usepackage{calc} % for calculating minipage widths
% Correct order of tables after \paragraph or \subparagraph
\usepackage{etoolbox}
\makeatletter
\patchcmd\longtable{\par}{\if@noskipsec\mbox{}\fi\par}{}{}
\makeatother
% Allow footnotes in longtable head/foot
\IfFileExists{footnotehyper.sty}{\usepackage{footnotehyper}}{\usepackage{footnote}}
\makesavenoteenv{longtable}
\usepackage{graphicx}
\makeatletter
\newsavebox\pandoc@box
\newcommand*\pandocbounded[1]{% scales image to fit in text height/width
  \sbox\pandoc@box{#1}%
  \Gscale@div\@tempa{\textheight}{\dimexpr\ht\pandoc@box+\dp\pandoc@box\relax}%
  \Gscale@div\@tempb{\linewidth}{\wd\pandoc@box}%
  \ifdim\@tempb\p@<\@tempa\p@\let\@tempa\@tempb\fi% select the smaller of both
  \ifdim\@tempa\p@<\p@\scalebox{\@tempa}{\usebox\pandoc@box}%
  \else\usebox{\pandoc@box}%
  \fi%
}
% Set default figure placement to htbp
\def\fps@figure{htbp}
\makeatother
% definitions for citeproc citations
\NewDocumentCommand\citeproctext{}{}
\NewDocumentCommand\citeproc{mm}{%
  \begingroup\def\citeproctext{#2}\cite{#1}\endgroup}
\makeatletter
 % allow citations to break across lines
 \let\@cite@ofmt\@firstofone
 % avoid brackets around text for \cite:
 \def\@biblabel#1{}
 \def\@cite#1#2{{#1\if@tempswa , #2\fi}}
\makeatother
\newlength{\cslhangindent}
\setlength{\cslhangindent}{1.5em}
\newlength{\csllabelwidth}
\setlength{\csllabelwidth}{3em}
\newenvironment{CSLReferences}[2] % #1 hanging-indent, #2 entry-spacing
 {\begin{list}{}{%
  \setlength{\itemindent}{0pt}
  \setlength{\leftmargin}{0pt}
  \setlength{\parsep}{0pt}
  % turn on hanging indent if param 1 is 1
  \ifodd #1
   \setlength{\leftmargin}{\cslhangindent}
   \setlength{\itemindent}{-1\cslhangindent}
  \fi
  % set entry spacing
  \setlength{\itemsep}{#2\baselineskip}}}
 {\end{list}}
\usepackage{calc}
\newcommand{\CSLBlock}[1]{\hfill\break\parbox[t]{\linewidth}{\strut\ignorespaces#1\strut}}
\newcommand{\CSLLeftMargin}[1]{\parbox[t]{\csllabelwidth}{\strut#1\strut}}
\newcommand{\CSLRightInline}[1]{\parbox[t]{\linewidth - \csllabelwidth}{\strut#1\strut}}
\newcommand{\CSLIndent}[1]{\hspace{\cslhangindent}#1}

\KOMAoption{captions}{tableheading}
\makeatletter
\@ifpackageloaded{caption}{}{\usepackage{caption}}
\AtBeginDocument{%
\ifdefined\contentsname
  \renewcommand*\contentsname{Table of contents}
\else
  \newcommand\contentsname{Table of contents}
\fi
\ifdefined\listfigurename
  \renewcommand*\listfigurename{List of Figures}
\else
  \newcommand\listfigurename{List of Figures}
\fi
\ifdefined\listtablename
  \renewcommand*\listtablename{List of Tables}
\else
  \newcommand\listtablename{List of Tables}
\fi
\ifdefined\figurename
  \renewcommand*\figurename{Figure}
\else
  \newcommand\figurename{Figure}
\fi
\ifdefined\tablename
  \renewcommand*\tablename{Table}
\else
  \newcommand\tablename{Table}
\fi
}
\@ifpackageloaded{float}{}{\usepackage{float}}
\floatstyle{ruled}
\@ifundefined{c@chapter}{\newfloat{codelisting}{h}{lop}}{\newfloat{codelisting}{h}{lop}[chapter]}
\floatname{codelisting}{Listing}
\newcommand*\listoflistings{\listof{codelisting}{List of Listings}}
\makeatother
\makeatletter
\makeatother
\makeatletter
\@ifpackageloaded{caption}{}{\usepackage{caption}}
\@ifpackageloaded{subcaption}{}{\usepackage{subcaption}}
\makeatother

\usepackage{bookmark}

\IfFileExists{xurl.sty}{\usepackage{xurl}}{} % add URL line breaks if available
\urlstyle{same} % disable monospaced font for URLs
\hypersetup{
  pdftitle={DSCI 310: Historical Horse Population in Canada},
  pdfauthor={Tiffany Timbers \& Jordan Bourak},
  colorlinks=true,
  linkcolor={blue},
  filecolor={Maroon},
  citecolor={Blue},
  urlcolor={Blue},
  pdfcreator={LaTeX via pandoc}}


\title{DSCI 310: Historical Horse Population in Canada}
\author{Tiffany Timbers \& Jordan Bourak}
\date{}

\begin{document}
\maketitle


\begin{verbatim}
Warning: package 'purrr' was built under R version 4.3.3
\end{verbatim}

\begin{verbatim}
-- Attaching core tidyverse packages ------------------------ tidyverse 2.0.0 --
v dplyr     1.1.4     v readr     2.1.5
v forcats   1.0.0     v stringr   1.5.1
v ggplot2   3.5.1     v tibble    3.2.1
v lubridate 1.9.3     v tidyr     1.3.1
v purrr     1.0.4     
-- Conflicts ------------------------------------------ tidyverse_conflicts() --
x dplyr::filter() masks stats::filter()
x dplyr::lag()    masks stats::lag()
i Use the conflicted package (<http://conflicted.r-lib.org/>) to force all conflicts to become errors
\end{verbatim}

\section{Aim}\label{aim}

This project explores the historical population of horses in Canada
between 1906 and 1972 for each province.

\section{Data}\label{data}

Horse population data were sourced from the
\href{http://open.canada.ca/en/open-data}{Government of Canada's Open
Data website} (Government of Canada, 2017a and Government of Canada,
2017b).

\section{Methods}\label{methods}

The R programming language (R Core Team, 2019) and the following R
packages were used to perform the analysis: knitr (Xie 2014), tidyverse
(Wickham 2017), and Quarto (Allaire et al 2022). \emph{Note: this report
is adapted from Timbers (2020).}

\section{Results}\label{results}

\begin{figure}

\centering{

\includegraphics[width=6.25in,height=\textheight,keepaspectratio]{../results/horse_pops_plot.png}

}

\caption{\label{fig-horse-pop}Horse populations for all provinces in
Canada from 1906--1972.}

\end{figure}%

Figure~\ref{fig-horse-pop}: Horse populations for all provinces in
Canada from 1906 - 1972.

We can see from Figure~\ref{fig-horse-pop} that Ontario, Saskatchewan
and Alberta have had the highest horse populations in Canada. All
provinces have had a decline in horse populations since 1940. This is
likely due to the rebound of the Canadian automotive industry after the
Great Depression and the Second World War. An interesting follow-up
visualisation would be car sales per year for each Province over the
time period visualised above to further support this hypothesis.

Suppose we were interested in looking in more closely at the province
with the highest spread (in terms of standard deviation) of horse
populations. We present the standard deviations in Table
Table~\ref{tbl-horse-sd}.

Table Table~\ref{tbl-horse-sd}. Standard deviation of historical
(1906-1972) horse populations for each Canadian province.

\begin{verbatim}
Rows: 9 Columns: 2
-- Column specification --------------------------------------------------------
Delimiter: ","
chr (1): Province
dbl (1): Std

i Use `spec()` to retrieve the full column specification for this data.
i Specify the column types or set `show_col_types = FALSE` to quiet this message.
\end{verbatim}

\begin{longtable}[]{@{}lr@{}}
\caption{Standard deviation of historical (1906--1972) horse populations
for each Canadian province.}\label{tbl-horse-sd}\tabularnewline
\toprule\noalign{}
Province & Std \\
\midrule\noalign{}
\endfirsthead
\toprule\noalign{}
Province & Std \\
\midrule\noalign{}
\endhead
\bottomrule\noalign{}
\endlastfoot
Saskatchewan & 377265.58 \\
Ontario & 266435.32 \\
Alberta & 266063.19 \\
Manitoba & 122403.87 \\
Quebec & 111411.10 \\
New Brunswick & 22019.49 \\
Nova Scotia & 19879.25 \\
British Columbia & 14945.66 \\
P.E.I. & 11355.75 \\
\end{longtable}

Note that we define standard deviation (of a sample) as

\[s = \sqrt{\frac{\sum_{i=1}^N (x_i - \overline{x})^2}{N-1} }\]

Additionally, note that in Table Table~\ref{tbl-horse-sd} we consider
the sample standard deviation of the number of horses during the same
time span as Figure~\ref{fig-horse-pop}.

\begin{figure}

\centering{

\includegraphics[width=4.375in,height=\textheight,keepaspectratio]{../results/horse_pop_plot_largest_sd.png}

}

\caption{\label{fig-horse-sd}Horse populations for the province with the
largest standard deviation.}

\end{figure}%

Figure~\ref{fig-horse-sd}: Horse populations for the province with the
largest standard deviation

In Figure~\ref{fig-horse-sd} we zoom in and look at the province of
Saskatchewan, which had the largest spread of values in terms of
standard deviation.

\section{References}\label{references}

R Core Team (2019) Timbers (2020) Allaire et al. (2022) Wickham (2017)
Xie (2014) Government of Canada (2017a) Government of Canada (2017b)

\phantomsection\label{refs}
\begin{CSLReferences}{1}{0}
\bibitem[\citeproctext]{ref-Allaire_Quarto_2022}
Allaire, J. J., Charles Teague, Carlos Scheidegger, Yihui Xie, and
Christophe Dervieux. 2022. {``{Quarto}.''}
\url{https://doi.org/10.5281/zenodo.5960048}.

\bibitem[\citeproctext]{ref-horses1}
Government of Canada. 2017a. {``Horses, Number on Farms at June 1 and at
December 1.''} Open Government - Open Data.
\url{https://open.canada.ca/data/en/dataset/a3ecf553-8ec4-4551-a0fe-8df1472c6cf7}.

\bibitem[\citeproctext]{ref-horses2}
---------. 2017b. {``Horses, Number on Farms at June 1, Farm Value Per
Head and Total Farm Value.''} Open Government - Open Data.
\url{https://open.canada.ca/data/en/dataset/e175ef9c-98f0-49b3-8131-ca0e3895a0cb}.

\bibitem[\citeproctext]{ref-R}
R Core Team. 2019. \emph{R: A Language and Environment for Statistical
Computing}. Vienna, Austria: R Foundation for Statistical Computing.
\url{https://www.R-project.org/}.

\bibitem[\citeproctext]{ref-ttimbers-horses}
Timbers, Tiffany. 2020. \emph{Historical Horse Population in Canada}.
\url{https://github.com/ttimbers/equine_numbers_value_canada_parameters}.

\bibitem[\citeproctext]{ref-tidyverse}
Wickham, Hadley. 2017. \emph{Tidyverse: Easily Install and Load the
'Tidyverse'}. \url{https://CRAN.R-project.org/package=tidyverse}.

\bibitem[\citeproctext]{ref-knitr}
Xie, Yihui. 2014. {``Knitr: A Comprehensive Tool for Reproducible
Research in {R}.''} In \emph{Implementing Reproducible Computational
Research}, edited by Victoria Stodden, Friedrich Leisch, and Roger D.
Peng. Chapman; Hall/CRC.
\url{http://www.crcpress.com/product/isbn/9781466561595}.

\end{CSLReferences}




\end{document}
